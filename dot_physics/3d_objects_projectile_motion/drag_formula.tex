\documentclass{article}
\usepackage{graphicx} % Required for inserting images
\usepackage[utf8]{inputenc}
\usepackage{amsmath, amssymb, amsthm}
\usepackage[margin = 1in]{geometry}
\setlength{\parindent}{0pt}
\title{Deriving Drag Force in Python Computable Terms}
\date{}


\begin{document}
\maketitle
\vspace{-2cm}
\begin{center}
    The goal is to convert the standard quadratic drag formula into terms that can be easily used in a python program, without having to separately update velocity 
\end{center}

The drag equation is:
\begin{equation}
    F_D = -\frac{1}{2} \rho ACv^2
\end{equation}

Where:

\hspace{1cm}$F_D$ is the drag force

\hspace{1cm}$\rho$ is the density of the fluid

\hspace{1cm}$A$ is the cross sectional area

\hspace{1cm}$C$ is the drag coefficient of the object

\hspace{1cm}$v$ is the velocity

\vspace{\baselineskip}

Vectorizing:
\begin{equation}
    \vec{F_D} = -\frac{1}{2} \rho AC |\vec{v}|^2\hat{v}
\end{equation}
\vspace{\baselineskip}

Substitution:

\hspace{1cm}Knowing that
\begin{equation}
    v = \frac{P}{m}
\end{equation}

Where:

\hspace{1cm}$P$ is the momentum
    
\hspace{1cm}$m$ is the mass
    
\hspace{1cm}$v$ is the velocity

\begin{equation}
    \vec{F_D} = \frac{1}{2} \rho AC \vec{\frac{|P|}{m}}^2 \hat{v}
\end{equation}

\vspace{\baselineskip}
Because momentum and velocity are in the same direction
($\vec{P}$ = m$\vec{v})$

\begin{equation}
    \vec{F_D} = -\frac{1}{2} \rho AC \vec{\frac{|P|}{m}}^2 \hat{P}
\end{equation}

\vspace{\baselineskip}

We now have an equation for drag force that we can use without having to calculate velocity and momentum separately.


\end{document}
